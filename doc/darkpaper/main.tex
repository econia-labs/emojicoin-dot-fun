\documentclass[table, twocolumn]{article}
\usepackage{amsmath}
\usepackage{geometry}
\usepackage[acronym]{glossaries}
\usepackage{pgfplots}
\usepackage{xcolor}
\pgfplotsset{compat=1.18}
\usetikzlibrary{arrows.meta}
\usetikzlibrary{intersections}

% Page options.
\pagecolor{black}
\color{gray}
\geometry{left=50pt, top=50pt, bottom=50pt, right=50pt}

% Tangent lines on graph per https://tex.stackexchange.com/a/198046.
\makeatletter
\def\parsenode[#1]#2\pgf@nil{%
  \tikzset{label node/.style = {#1}}
\def\nodetext{#2}
}
\tikzset{add node at x/.style 2 args = {name path global = plot line,
      /pgfplots/execute at end plot visualization/.append = {\begingroup
          \@ifnextchar[{\parsenode}{\parsenode[]}#2\pgf@nil
          \path [name path global = position line #1-1]
          ({axis cs:#1,0}|-{rel axis cs:0,0}) --
          ({axis cs:#1,0}|-{rel axis cs:0,1});
          \path [xshift = 1pt, name path global = position line #1-2]
          ({axis cs:#1,0}|-{rel axis cs:0,0}) --
          ({axis cs:#1,0}|-{rel axis cs:0,1});
          \path [
            name intersections =
              {of = {plot line and position line #1-1}, name = left intersection},
            name intersections =
              {of = {plot line and position line #1-2}, name = right intersection},
            label node/.append style = {pos = 1}
          ] (left intersection-1) -- (right intersection-1)
          node [label node]{\nodetext};
          \endgroup
        }
    }
}
\makeatother


% Acronyms.
\newacronym{cpamm}{CPAMM}{Constant Product Automated Market Maker}
\newacronym{clamm}{CLAMM}{Concentrated Liquidity Automated Market Maker}

\title{emojicoin dot fun}
\author{Econia Labs}
\date{}

\begin{document}

\maketitle

\section{Bonding curve}

The state transition occurs at pre-defined nominal market capitalization $m_n$, for
example 10,000 \texttt{APT}. Internally this corresponds to a market capitalization in
octas $m_o$ as defined in equation \ref{eqn:state-transition-market-cap}.

\begin{equation} \label{eqn:state-transition-market-cap}
  m_o = 10^8 m_n
\end{equation}

\begin{table}[!htb]
  \centering
  \begin{tabular}{|c|c|}
    \hline \rowcolor{blue}
    Term                           & Notation  \\ \hline
    Real base reserves             & $b_r$     \\ \hline
    Real quote reserves            & $q_r$     \\ \hline
    Virtual base reserves          & $b_v$     \\ \hline
    Virtual quote reserves         & $q_v$     \\ \hline
    Liquidity                      & $L$       \\ \hline
    Low price range endpoint       & $p_l$     \\ \hline
    High price range endpoint      & $p_h$     \\ \hline
    Real base reserves ceiling     & $b_{r,c}$ \\ \hline
    Real quote reserves ceiling    & $q_{r,c}$ \\ \hline
    Virtual base reserves ceiling  & $b_{v,c}$ \\ \hline
    Virtual quote reserves ceiling & $q_{v,c}$ \\ \hline
    Virtual base reserves floor    & $b_{v,f}$ \\ \hline
    Virtual quote reserves floor   & $q_{v,f}$ \\ \hline
  \end{tabular}
  \caption{Terms, translation from \gls{cpamm} to \gls{clamm}}
  \label{tab:clamm-curve-translation}
\end{table}

\begin{equation} \label{eqn:clamm-curve-translation}
  (b_r + b_{v, f})(q_r + q_{v, f}) = L^2 = b_v q_v
\end{equation}

\begin{figure}[!htb]
  \centering
  \begin{tikzpicture}
  \begin{axis}[
      axis lines = left,
      xlabel = Emojicoin reserves,
      ylabel = \texttt{APT} reserves,
      xmin = 0,
      xmax = 3,
      ymin = 0,
      ymax = 3,
      ytick=\empty,
      xtick=\empty,
      extra x ticks = {0.5, 1.5, 2},
      extra x tick labels = {$b_{v, f}$, $b_{r,c}$, $b_{v, c}$},
      extra y ticks = {0.5, 1.5, 2},
      extra y tick labels = {$q_{v, f}$, $q_{r,c}$, $q_{v, c}$},
      tick style = {thick, major tick length = 7pt},
      legend style = {fill = black, draw = gray},
      % Tangent lines on graph per https://tex.stackexchange.com/a/198046.
      tangent/.style={add node at x={2}{[
                  sloped, minimum width = 75pt,
                  append after command =
                    {(\tikzlastnode.west) edge [thick] (\tikzlastnode.east)}
                ]},
          add node at x={0.5}{[
                  sloped, minimum width = 75pt,
                  append after command =
                    {(\tikzlastnode.west) edge [thick] (\tikzlastnode.east)}
                ]}
        }
    ]
    \addplot [
      domain = 0:5,
      samples = 100,
      color = blue,
      thick,
      tangent,
    ] {1 / x};
    \addlegendentry{Virtual reserves}
    \addplot [
      domain = 0:5,
      samples = 100,
      color = green,
      thick,
    ] {1 / (x + 0.5) - 0.5};
    \addlegendentry{Real reserves}
    \node at (2.3, 0.75) [] {$p_{l} = -\frac{dq_v}{db_v}|_{b_{v,c}} = \frac{q_{v, f}}{b_{v, c}}$};
    \node at (1.5, 0.175) [] {$p_{l}$} ;
    \node at (1.35, 2) [] {$p_{h} = -\frac{dq_v}{db_v}|_{b_{v,f}} = \frac{q_{v, c}}{b_{v, f}}$};
    \node at (0.175, 1.5) [] {$p_{h}$} ;
    \node at (2, 0.5) [circle, fill, scale = 0.5] {};
    \node at (0.5, 2) [circle, fill, scale = 0.5] {};
    \draw [dashed] (2, 0) -- (2, 0.5);
    \draw [dashed] (0, 2) -- (0.5, 2);
    \draw [dashed] (0, 0.5) -- (2, 0.5);
    \draw [dashed] (0.5, 0) -- (0.5, 2);
    \draw [arrows = {-Latex[]}] (2, 0.5) -- (1.5, 0);
    \draw [arrows = {-Latex[]}] (0.5, 2) -- (0, 1.5);
  \end{axis}
\end{tikzpicture}

  \caption{Translation from \gls{cpamm} to \gls{clamm}}
  \label{fig:clamm-curve-translation}
\end{figure}

With the state transition constrained by $m_n$, it follows that $q_{r, c} = m_o$. Hence
either a circulating supply at the time of transition ($b_{r, c}$) or a price at the
time of state transition can be chosen to fully constrain the bonding curve ($p_h$)

there is thus a relationship between
nominal number of circulating emojicoins and nominal price at the time of the state
transition per equation \ref{eqn:state-transition-relationship}
nominal amount

\begin{equation} \label{eqn:state-transition-relationship}
  b_{r, c} = \frac{m_n}{p_h}
\end{equation}


At the state transition, the price is emojicoin priced at 1 \texttt{APT} per, such that
$p_h = 1$. Additionally,

\begin{equation} \label{eqn:b-q-out-clamm-simple}
  b_{out} = \frac{q_{in} b_{v, 0}}{q_{in} + q_{v, 0}},
  q_{out} = \frac{b_{in} q_{v, 0}}{b_{in} + b_{v, 0}}
\end{equation}

\end{document}
