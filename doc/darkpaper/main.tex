\documentclass[table, twocolumn]{article}
\usepackage{amsmath}
\usepackage{geometry}
\usepackage[acronym]{glossaries}
\usepackage{pgfplots}
\usepackage{xcolor}
\pgfplotsset{compat=1.18}
\usetikzlibrary{arrows.meta}
\usetikzlibrary{intersections}

% Page options.
\pagecolor{black}
\color{gray}
\geometry{left=50pt, top=50pt, bottom=50pt, right=50pt}

% Tangent lines on graph per https://tex.stackexchange.com/a/198046.
\makeatletter
\def\parsenode[#1]#2\pgf@nil{%
  \tikzset{label node/.style = {#1}}
\def\nodetext{#2}
}
\tikzset{add node at x/.style 2 args = {name path global = plot line,
      /pgfplots/execute at end plot visualization/.append = {\begingroup
          \@ifnextchar[{\parsenode}{\parsenode[]}#2\pgf@nil
          \path [name path global = position line #1-1]
          ({axis cs:#1,0}|-{rel axis cs:0,0}) --
          ({axis cs:#1,0}|-{rel axis cs:0,1});
          \path [xshift = 1pt, name path global = position line #1-2]
          ({axis cs:#1,0}|-{rel axis cs:0,0}) --
          ({axis cs:#1,0}|-{rel axis cs:0,1});
          \path [
            name intersections =
              {of = {plot line and position line #1-1}, name = left intersection},
            name intersections =
              {of = {plot line and position line #1-2}, name = right intersection},
            label node/.append style = {pos = 1}
          ] (left intersection-1) -- (right intersection-1)
          node [label node]{\nodetext};
          \endgroup
        }
    }
}
\makeatother


% Acronyms.
\newacronym{cpamm}{CPAMM}{Constant Product Automated Market Maker}
\newacronym{clamm}{CLAMM}{Concentrated Liquidity Automated Market Maker}

\title{emojicoin dot fun}
\author{Econia Labs}
\date{}

\begin{document}

\maketitle

\section{State transition}

The state transition occurs at pre-defined nominal market capitalization in
\texttt{APT} ($m_a$), for example 10,000 \texttt{APT}. This corresponds to a market
capitalization in octas $m_o$ as defined in equation
\ref{eqn:state-transition-market-cap-apt}.

\begin{equation} \label{eqn:state-transition-market-cap-apt}
  m_o = 10^8 m_a
\end{equation}

At the state transition, the spot price resolves to $p_{crit}$, for example 1
\texttt{APT} per emojicoin.

When the price finally reaches $p_{crit}$, there are $m_s$ emojicoin subunits in
circulation. Taking emojicoin decimals as identical to \texttt{APT} decimals, $m_s$
relates to nominal emojicoin units $m_e$ as defined in equation
\ref{eqn:state-transition-market-cap-emojicoin}.

\begin{equation} \label{eqn:state-transition-market-cap-emojicoin}
  m_s = 10^8 m_e
\end{equation}

\section{Bonding curve}

The bonding curve is represented by a \gls{clamm}, which functions as a \gls{cpamm}
within a specific range as defined by table \ref{tab:clamm-curve-translation}, equation
\ref{eqn:clamm-curve-translation}, and figure \ref{fig:clamm-curve-translation}.

\begin{table}[!htb]
  \centering
  \begin{tabular}{|c|c|}
    \hline \rowcolor{blue}
    Term                           & Notation  \\ \hline
    Real base reserves             & $b_r$     \\ \hline
    Real quote reserves            & $q_r$     \\ \hline
    Virtual base reserves          & $b_v$     \\ \hline
    Virtual quote reserves         & $q_v$     \\ \hline
    Liquidity                      & $L$       \\ \hline
    Low price range endpoint       & $p_l$     \\ \hline
    High price range endpoint      & $p_h$     \\ \hline
    Real base reserves ceiling     & $b_{r,c}$ \\ \hline
    Real quote reserves ceiling    & $q_{r,c}$ \\ \hline
    Virtual base reserves ceiling  & $b_{v,c}$ \\ \hline
    Virtual quote reserves ceiling & $q_{v,c}$ \\ \hline
    Virtual base reserves floor    & $b_{v,f}$ \\ \hline
    Virtual quote reserves floor   & $q_{v,f}$ \\ \hline
  \end{tabular}
  \caption{Terms, translation from \gls{cpamm} to \gls{clamm}}
  \label{tab:clamm-curve-translation}
\end{table}

\begin{equation} \label{eqn:clamm-curve-translation}
  (b_r + b_{v, f})(q_r + q_{v, f}) = L^2 = b_v q_v
\end{equation}

\begin{figure}[!htb]
  \centering
  \begin{tikzpicture}
  \begin{axis}[
      axis lines = left,
      xlabel = Emojicoin reserves,
      ylabel = \texttt{APT} reserves,
      xmin = 0,
      xmax = 3,
      ymin = 0,
      ymax = 3,
      ytick=\empty,
      xtick=\empty,
      extra x ticks = {0.5, 1.5, 2},
      extra x tick labels = {$b_{v, f}$, $b_{r,c}$, $b_{v, c}$},
      extra y ticks = {0.5, 1.5, 2},
      extra y tick labels = {$q_{v, f}$, $q_{r,c}$, $q_{v, c}$},
      tick style = {thick, major tick length = 7pt},
      legend style = {fill = black, draw = gray},
      % Tangent lines on graph per https://tex.stackexchange.com/a/198046.
      tangent/.style={add node at x={2}{[
                  sloped, minimum width = 75pt,
                  append after command =
                    {(\tikzlastnode.west) edge [thick] (\tikzlastnode.east)}
                ]},
          add node at x={0.5}{[
                  sloped, minimum width = 75pt,
                  append after command =
                    {(\tikzlastnode.west) edge [thick] (\tikzlastnode.east)}
                ]}
        }
    ]
    \addplot [
      domain = 0:5,
      samples = 100,
      color = blue,
      thick,
      tangent,
    ] {1 / x};
    \addlegendentry{Virtual reserves}
    \addplot [
      domain = 0:5,
      samples = 100,
      color = green,
      thick,
    ] {1 / (x + 0.5) - 0.5};
    \addlegendentry{Real reserves}
    \node at (2.3, 0.75) [] {$p_{l} = -\frac{dq_v}{db_v}|_{b_{v,c}} = \frac{q_{v, f}}{b_{v, c}}$};
    \node at (1.5, 0.175) [] {$p_{l}$} ;
    \node at (1.35, 2) [] {$p_{h} = -\frac{dq_v}{db_v}|_{b_{v,f}} = \frac{q_{v, c}}{b_{v, f}}$};
    \node at (0.175, 1.5) [] {$p_{h}$} ;
    \node at (2, 0.5) [circle, fill, scale = 0.5] {};
    \node at (0.5, 2) [circle, fill, scale = 0.5] {};
    \draw [dashed] (2, 0) -- (2, 0.5);
    \draw [dashed] (0, 2) -- (0.5, 2);
    \draw [dashed] (0, 0.5) -- (2, 0.5);
    \draw [dashed] (0.5, 0) -- (0.5, 2);
    \draw [arrows = {-Latex[]}] (2, 0.5) -- (1.5, 0);
    \draw [arrows = {-Latex[]}] (0.5, 2) -- (0, 1.5);
  \end{axis}
\end{tikzpicture}

  \caption{Translation from \gls{cpamm} to \gls{clamm}}
  \label{fig:clamm-curve-translation}
\end{figure}

As derived in section \ref{sec:bonding-curve-constraints}, the bonding curve initializes
with virtual reserves defined in equation \ref{eqn:bonding-curve-setup}, ending when
virtual reserves reach the values from equation \ref{eqn:bonding-curve-transition}.

\begin{equation} \label{eqn:bonding-curve-setup}
  b_{v, c} = m_s + \frac{m_o}{p_{crit} - \frac{m_o}{m_s}},
  q_{v, f} = \frac{m_o}{\frac{m_s}{m_o} \cdot p_{crit} - 1}
\end{equation}

\begin{equation} \label{eqn:bonding-curve-transition}
  b_{v, f} = \frac{m_o}{p_{crit} - \frac{m_o}{m_s}},
  q_{v, c} = m_o + \frac{m_o}{\frac{m_s}{m_o} \cdot p_{crit} - 1}
\end{equation}

Throughout the bonding curve, the virtual reserve amounts in the \gls{clamm} follow a
simple constant product curve invariant, yielding equation
\ref{eqn:b-q-out-cpamm-simple} as derived in section
\ref{eqn:b-q-out-simple-derivation}.

\begin{equation} \label{eqn:b-q-out-cpamm-simple}
  b_{out} = \frac{q_{in} b_{v, 0}}{q_{in} + q_{v, 0}},
  q_{out} = \frac{b_{in} q_{v, 0}}{b_{in} + b_{v, 0}}
\end{equation}

\section{Derivations}

\subsection{Bonding curve constraints} \label{sec:bonding-curve-constraints}

Evaluated at $p_l$, (\ref{eqn:clamm-curve-translation}) reduces to:

\begin{equation} \label{eqn:bonding-curve-1}
  (b_{r, c} + b_{v, f}) \cdot q_{v, f} = L^2
\end{equation}

Likewise, (\ref{eqn:clamm-curve-translation}) evaluated at $p_h$ reduces to:

\begin{equation} \label{eqn:bonding-curve-2}
  b_{v, f} \cdot (q_{r, c} + q_{v, f}) = L^2
\end{equation}

For a fixed $b_{r, c} = m_s$ and $q_{r, c} = m_o$ combining (\ref{eqn:bonding-curve-1})
and (\ref{eqn:bonding-curve-2}) yields:

\begin{align} \label{eqn:bonding-curve-3}
  (b_{r, c} + b_{v, f}) \cdot q_{v, f}              & =
  b_{v, f} \cdot (q_{r, c} + q_{v, f}) \nonumber              \\
  b_{r, c} \cdot q_{v, f} + b_{v, f} \cdot q_{v, f} & =
  b_{v, f} \cdot q_{r, c} + b_{v, f} \cdot q_{v, f} \nonumber \\
  b_{r, c} \cdot q_{v, f}                           & =
  b_{v, f} \cdot q_{r, c} \nonumber                           \\
  \frac{b_{r, c}}{q_{r, c}} q_{v, f}                & =
  b_{v, f} \nonumber                                          \\
  b_{v, f}                                          & =
  \frac{b_{r, c}}{q_{r, c}} \cdot q_{v, f} \nonumber          \\
  b_{v, f}                                          & =
  \frac{m_s}{m_o} \cdot q_{v, f}
\end{align}

$p_h = p_{crit}$ yields:

\begin{align} \label{eqn:bonding-curve-4}
  p_h                                                 & =
  \frac{q_{v, c}}{b_{v, f}} \nonumber                      \\
  p_h                                                 & =
  \frac{q_{r, c} + q_{v, f}}{b_{v, f}} \nonumber           \\
  p_h \cdot b_{v, f}                                  & =
  q_{r, c} + q_{v, f} \nonumber                            \\
  \frac{m_e}{m_a} \cdot p_h \cdot q_{v, f}            & =
  q_{r, c} + q_{v, f} \nonumber                            \\
  \frac{m_e}{m_a} \cdot p_h \cdot q_{v, f} - q_{v, f} & =
  q_{r, c} \nonumber                                       \\
  q_{v, f} (\frac{m_s}{m_0} \cdot p_h - 1)            & =
  q_{r, c} \nonumber                                       \\
  q_{v, f}                                            & =
  \frac{q_{r, c}}{\frac{m_s}{m_o} \cdot p_h - 1} \nonumber \\
  q_{v, f}                                            & =
  \frac{m_o}{\frac{m_s}{m_o} \cdot p_{crit} - 1}
\end{align}

Substituting (\ref{eqn:bonding-curve-4}) into (\ref{eqn:bonding-curve-3}) yields:

\begin{align} \label{eqn:bonding-curve-5}
  b_{v, f} & =
  \frac{m_s}{m_o} \cdot \frac{m_o}{\frac{m_s}{m_o} \cdot p_{crit} - 1} \nonumber \\
  b_{v, f} & =
  \frac{m_s \cdot m_o}{m_s \cdot p_{crit} - m_o} \nonumber                       \\
  b_{v, f} & = \frac{m_o}{p_{crit} - \frac{m_o}{m_s}}
\end{align}

Hence the \gls{clamm} is initialized with $q_{v, f}$ per (\ref{eqn:bonding-curve-4}) and
$b_{v, c}$ per:

\begin{align}
  b_{v, c} & = b_{r, c} + b_{v, f} \nonumber                \\
  b_{v, c} & = m_s + \frac{m_o}{p_{crit} - \frac{m_o}{m_s}}
\end{align}

The other bonding curve regime endpoint is thus at $b_{v, f}$ per
(\ref{eqn:bonding-curve-5}) and $q_{v, c}$ per:

\begin{align}
  q_{v, c} & = q_{r, c} + q_{v, f} \nonumber                        \\
  q_{v, c} & = m_o + \frac{m_o}{\frac{m_s}{m_o} \cdot p_{crit} - 1}
\end{align}

\subsection{Base in and out for \gls{cpamm} swap} \label{eqn:b-q-out-simple-derivation}

\begin{align}
  b_0 q_0               & = b_f q_f \nonumber                                      \\
  b_0 q_0               & = (b_0 - b_{out})(q_0 + q_{in}) \nonumber                \\
  b_0 q_0               & = b_0 q_0 + b_0 q_{in} - b_{out}(q_0 + q_{in}) \nonumber \\
  b_{out}(q_0 + q_{in}) & = b_{0} q_{in} \nonumber                                 \\
  b_{out}               & = \frac{q_{in} b_0}{q_{in} + q_0}
\end{align}

\begin{align}
  b_0 q_0               & = b_f q_f \nonumber                                      \\
  b_0 q_0               & = (b_0 + b_{in})(q_0 - q_{out}) \nonumber                \\
  b_0 q_0               & = b_0 q_0 + b_{in} q_0 - q_{out}(b_0 + b_{in}) \nonumber \\
  q_{out}(b_0 + b_{in}) & = b_{in} q_0 \nonumber                                   \\
  q_{out}               & = \frac{b_{in} q_0}{b_{in} + b_0}
\end{align}


\end{document}
